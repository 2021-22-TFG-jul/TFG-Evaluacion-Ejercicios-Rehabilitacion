\capitulo{1}{Introducción}

En la enfermedad de \textit{Parkinson} los síntomas motores representan la base del diagnóstico. Uno de los principales síntomas es la \texttt{bradicinesia} \footnote{La bradicinesia es un trastorno caracterizado por la lentitud de movimientos} pero esta enfermedad también cuenta con otros síntomas no motores como pueden ser trastornos del sueño, sensoriales y neuropsiquiátricos como la depresión ~\cite{enrique2019latencia} ~\cite{Parkinson1}.

Es una enfermedad que afecta sobretodo a la población de avanzada edad con un promedio de inicio a partir de los 60 años y siendo dos veces más frecuente en hombres que en mujeres. Aunque cabe destacar que aun siendo menos frecuente esta patología en mujeres, una vez que la han desarrollado, experimentan una tasa de supervivencia más baja y una progresión de la enfermedad mucho mayor padeciendo unos síntomas no motores más graves que los de los hombres \cite{ortiz2020diferencias}. 

Actualmente, no hay cura para la enfermedad de \textit{Parkinson}. Por ello la fisioterapia desempeña un papel crucial en el manejo de los síntomas motores de los pacientes que padecen esta enfermedad. Para poder llevar un seguimiento exhaustivo de la precisión con la que se realizan los ejercicios prescritos, los pacientes han de estar acompañados por un terapeuta, lo que conlleva un desplazamiento de su domicilio varias veces por semana ~\cite{saba2022guidelines}. Contando con que los pacientes suelen ser personas mayores y muchas veces dependientes, para una gran cantidad de familias es complejo adaptarse a una rutina en la que tienen que trasladar o acompañar al paciente hasta su centro de salud más cercano. En muchas ocasiones son familias que residen en el medio rural y necesitan hacer uso del trasporte privado para poder acudir a las visitas del terapeuta. 
Estas visitas deben hacerse periódicamente y esta es una de las consecuencias de la saturación de los servicios. Desgraciadamente no existe suficiente personal sanitario que pueda atender a los pacientes con la frecuencia que necesitan, o incluso en ocasiones puede que las sesiones sean con diferentes terapeutas, imposibilitando un buen seguimiento de la progresión de la enfermedad ~\cite{garcia2016informe}. 

Por estas razones se pretende crear un sistema por el cual los pacientes puedan realizar los ejercicios cómodamente en su hogar y a través de un análisis posterior determinar si se han realizado correctamente.

Este proyecto surge a partir de un par de trabajos fin de máster creados por los compañeros \textit{José Luis Garrido Labrador} y \textit{José Miguel Ramírez Sanz}. En sus proyectos se ofrece un programa por el cual se obtiene una descomposición en imágenes a partir de un vídeo de referencia. Estas imágenes pueden ser \texttt{anonimizadas}\footnote{ Expresar un dato relativo a entidades o personas, eliminando la referencia a su identidad.} para ocultar el rostro de la persona y sobretodo, una vez extraídas las imágenes se realiza una extracción del esqueleto del paciente. Este esqueleto puede ser almacenado como una secuencia, perteneciendo cada valor a una posición del esqueleto respecto de la imagen. 

El proyecto que se ha logrado es la búsqueda y comparación de estas secuencias que componen un esqueleto. A partir de todas, o una gran parte de las secuencias que el programa puede sacar de un vídeo completo, se dispondrán a ser almacenadas. Una vez que se tienen almacenadas tanto las secuencias de un ejercicio completo por una parte, como las secuencias de varios ejercicios concretos por otra, se ha implementado un algoritmo que localiza la secuencia que representa el ejercicio específico dentro de la la secuencia compuesta por varios ejercicios, con el objetivo de poder comparar entre entre el ejercicios de referencia que realiza el terapeuta con el mismo ejercicio pero en este caso realizado por el paciente. Hay que tener en cuenta que los pacientes pueden realizar vídeos a una velocidad inferior que el terapeuta, o que los movimientos realizados pueden ser parecidos pero nunca iguales. Todo ello se ha tenido en cuenta para las comparaciones de secuencias.

Finalmente una vez se haya detectado en que punto exacto comienza y termina un ejercicio, por medio de la aplicación de escritorio que ha sido creada para mostrar el funcionamiento del programa, se ha recortado el vídeo original y se ha implementado un código que muestra el ejercicio buscado al usuario. 

Una mejora muy interesante sería informar de como de bien o mal realizado está ese ejercicio pero más adelante se detallarán los inconvenientes que han surgido al intentar resolver este problema.  
