\capitulo{2}{Objetivos del proyecto}
\section{Objetivos generales}
\begin{itemize}
    \item Investigar como implantar el algoritmo \textit{dynamic time warping} en la búsqueda de secuencias.
    \item Realizar una exhaustiva investigación sobre los algoritmos de comparación de secuencias temporales, para comprobar el inicio y el final de éstas en otras secuencias.
    \item Investigar múltiples librerías en \textit{Python} que permitan la localización de secuencias multidimensionales lo más parecidas posible a un patrón de referencia, dentro de una secuencia de mayor tamaño.
    \item Investigar y analizar como se podría clasificar un ejercicio según las partes del cuerpo que estén en movimiento.
    \item Recopilación de vídeos adecuados para la investigación, tanto para ser empleados en este proyecto como en proyectos futuros, comprobando que son idóneos para su análisis. 
    \item Realizar una aplicación de escritorio por la cual se pueda observar un recorte de un ejercicio concreto de un vídeo con múltiples ejercicios.
\end{itemize}

\section{Objetivos técnicos}
\begin{itemize}
    \item Aplicar las técnicas investigadas a las secuencias de esqueletos.
    \item Desarrollar un algoritmo en \textit{Python} que permita la búsqueda y obtención de secuencias dentro de una de mayor tamaño.
    \item Desarrollar una web para mostrar el contenido de mi proyecto
    \item Aprender y poner en práctica el manejo de contenedores \textit{Docker}
    \item Utilizar un sistema de control de versiones, utilizando para ello la plataforma \textit{Github}.
    \item Utilizar el sistema de composición de textos \LaTeX{} para la realización de la memoria, anexos y conseguir una cierta destreza. 
    \item Utilizar el editor \textit{Overleaf} para la realización de la memoria y anexos.
\end{itemize}

\section{Objetivos personales}
En este apartado se van a exponer los objetivos personales que me he propuesto de cara a la realización del proyecto. 
\begin{itemize}
    \item Intentar aportar algún enfoque o técnica novedosa para la mejora de la calidad de vida de pacientes con la enfermedad de \textit{Parkinson}.
    \item Poner en práctica varios de los conocimientos adquiridos en la carrera.
    \item Investigar e indagar sobre las tecnologías \textit{Data Mining}.
    
\end{itemize}
